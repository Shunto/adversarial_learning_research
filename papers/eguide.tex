\documentstyle[master,english]{kuisthesis}

\def\LATEX{{\rm (L\kern-.36em\raise.3ex\hbox{\sc a})\TeX}}
\def\LATex{\iLATEX\small}
\def\iLATEX#1{L\kern-.36em\raise.3ex\hbox{#1\bf A}\kern-.15em
    T\kern-.1667em\lower.7ex\hbox{E}\kern-.125emX}
\def\LATEXe{\ifx\LaTeXe\undefined \LaTeX 2e\else\LaTeXe\fi}
\def\LATExe{\ifx\LaTeXe\undefined \iLATEX\scriptsize 2e\else\LaTeXe\fi}
\let\EM\bf
\def\|{\verb|}
\def\<{\(\langle\)}
\def\>{\(\rangle\)}
\def\CS#1{{\tt\string#1}}

\jtitle{$BFCJL8&5fJs9p=q!&=$;NO@J8<9I.$N<j0z(B}	% Title in Japanese
\etitle[How to Write Your B.E./M.E. Thesis]%	% Title for Abstract/TOC
	{How to Write\\Your B.E./M.E. Thesis}	% Title for title page
\jauthor{$BCfEg(B $B9@(B}				% Author name in Japanese
\eauthor{Hiroshi NAKASHIMA}			% Author name in English
\supervisor{Professor Toru Ishida}		% Name of your Supervisor
\date{Feburary 15, 1996}			% Submission date
\department{Social Informatics}			% Dept. name

\begin{document}
\maketitle					% Output title page

\begin{eabstract}				% Abstract in English
This guide gives instructions for writing your B.E. or M.E. theses following
the standard of the Department of Information Science.  The
standard includes the structure and format which you must obey on writing
your theses.

This guide also explains how to use a \LaTeX{} style file for theses, named
\verb|kuisthesis|, with which you can easily produce well-formatted results.
Since this guide itself is produced with the style file, it will help you to
refer its source file \verb|eguide.tex| as an example.

Note for graduate students: This document is written for students of
old graduate school of information science, not for graudate school of
informatics. Writers of master thesis belonging to graduate school
of informatics must obey rules given by each department.

\par\bigskip\centerline{\bf NOTE}
\begin{quote}
This guide is translated from ``$BFCJL8&5fJs9p!&=$;NO@J8:n@.$N<j0z(B'' in order
to provide an example written in English.  Although you may regard this
guide as giving the same guidelines and instructions which the Japanese
version gives, remember that the Japanese version is solely and formally
established by the Department.
\end{quote}
\end{eabstract}

\begin{jabstract}				% Abstract in Japanese
$B$3$N<j0z$G$O!$FCJL8&5fJs9p=q$*$h$S=$;NO@J8$r$I$N$h$&$J9=@.$H$9$k$+!$$^$?$I$N(B
$B$h$&$J7A<0$G:n@.$9$k$+$r@bL@$7$?$b$N$G$"$k!#$^$?!$Ev65<<$GDj$a$?7A<0$KB'$C$?(B
$BO@J8$rF|K\8l(B\LaTeX $B$rMQ$$$F:n@.$9$k$?$a$N%9%?%$%k!&%U%!%$%k$G$"$k!$(B 
\verb|kuisthesis|$B$N;H$$J}$K$D$$$F$b@bL@$7$F$$$k!#$J$*!$$3$N<j0z<+BN$b(B
\verb|kuisthesis|$B$rMQ$$!$Dj$a$i$l$?7A<0$K=>$C$F:n@.$5$l$F$$$k$N$G!$I,MW$K1~$8(B
$B$F%=!<%9!&%U%!%$%k(B\verb|eguide.tex|$B$r;2>H$5$l$?$$!#(B

\par\bigskip\centerline{\bf $BCm0U(B}
\begin{quote}
$B$3$N<j0z$O!$1QJ8O@J8$NNc$H$7$FMQ$$$k$?$a$KF|K\8lHG$N!VFCJL8&5fJs9p!&=$;NO@J8(B
$B:n@.$N<j0z!W$rK]Lu$7$?$b$N$G$"$k!#$3$N<j0z$K<($5$l$F$$$k;X<($J$I$O!$F|K\8lHG(B
$B$N$b$N$HF10l$G$"$k$H9M$($F$b$h$$$,!$@5<0$K$OF|K\8lHG$N$_$,65<<$GDj$a$i$l$?$b(B
$B$N$G$"$k$3$H$KCm0U$9$k$3$H!#(B
\end{quote}
\end{jabstract}

\tableofcontents				% Output table of contents

\section{Introduction}\label{sec-intro}		% Here the main text starts
As you graduate the B.E. or M.E. course, you have to submit your graduation
research report (aka B.E. thesis) or M.E. thesis in which the results of
your research work in the course are compiled.  Thus, your thesis should be
respected as an important milestone on finishing the course.  Your thesis
will be kept in the Department library long years so that teaching staffs
and student refer it.

Your thesis must describes your research work clearly keeping a given page
limit.  You must pay close and great attention to the structure of your
thesis and every sentences, phrases and words.  This guide shows guidelines
of writing theses and instructions to make theses meet the standard of the
Department.  Following these guidelines and instructions, however, are not
sufficient to write a good thesis, but they should be regarded as minimum
requirements.  The objective of writing a thesis is not to meet a standard
but to make readers understand your research works correctly and clearly.
Therefore, it is strongly recommended to read good research papers so that
you develop your faculty for description and expression.

In the following sections, the standard structure of a thesis is shown, and
then detailed instructions on writing a thesis are given.  In appendix, a
\LaTeX{} style file for theses, \verb|kuisthesis|, and how to use it are
explained.

\par\bigskip\centerline{\bf NOTE}
\begin{quote}
This guide is translated from ``$BFCJL8&5fJs9p!&=$;NO@J8:n@.$N<j0z(B'' in order
to provide an example written in English.  Although you may regard this
guide as giving the same guidelines and instructions which the Japanese
version gives, remember that the Japanese version is solely and formally
established by the Department.
\end{quote}

\section{Structure of Thesis}\label{sec-structure}
A thesis must consist of an abstract, a table of contents and the main text.
Appendices may be added if necessary.  Since each component has its own role,
a component must be separated clearly from others and must have description
appropriate to its role.

\subsection{Abstract}\label{subsec-abstract}
The abstract of a thesis summarizes what the research work aimed at and what
conclusion was formed.  Thus, the abstract is not a condensed version of the
thesis, but has to be essence of the thesis.  Also note that an abstract
is not an introduction.

The standard of the Department requires that a thesis must have both
Japanese and English abstracts, each of which is compiled in two pages,
irrespective of language in which the main text is written.

\subsection{Table of Contents}\label{subsec-toc}
The table of contents has a form like ordinary text books and is placed just
before the main text.  The table is useful to readers not only in looking up
a section, but also in grasping the structure of your thesis.

\subsection{Main Text}\label{subsec-main}
The main text should be self-contained.  Although your thesis may have
appendices, your thesis must be understandable by reading the main text
only.

It is strongly recommended that the main text has an introduction, the main
issue, and a conclusion.

\subsubsection{Introduction}\label{subsubsec-intro}
The introduction describes what position your research work occupies in a
research area, what the research aims at, and how the research is
characterized.   Through those descriptions, readers will grasp the
significance of your research work before reading the main issue.

Note that an introduction is different from an abstract in which the essence
of a thesis is described.  Rather, the introduction makes readers {\em
warmed-up} and ready to proceed to the main issue.  Thus, it may be
necessary to show the past and/or current state of the activities in the
research area as a background, and to mention the relationship and
difference between your work and other related works.

\subsubsection{Main Issue}\label{subsubsec-main}
Since the main issue is obviously the body of your thesis, most of pages of
the thesis should be given to the main issue.  You must make much account of
the completeness and consistency of the thesis, while emphasizing important
issues.  Also you must keep conscientious description in your mind, strictly
and clearly separating what you did and what others did.

In order to keep the clearness of your thesis, which is the most important
feature, take account of the following advice.
\begin{itemize}%{
\item
Split the main issue into chapters and sections lining up them appropriately
and giving them good titles.  A section may be split further into smaller
pieces that should have appropriate titles too.
\item
Make each chapter and section complete, and keep natural relationship
between them.
\item
Use simple and concise expressions.  Emphasize important and/or original
issues describing them in detail, while summarize other parts.
\item
Define meaning of symbols used in your thesis clearly and accurately.
Arrange the order and detailedness of formulae in a derivation carefully.
If a derivation has long and complicated formulae, leave some important ones
in the main text, while moving others to an appendix.
\item
Show bases of approximated and/or experimental equations, if any, clearly.
\item
Use figures and tables, which are often very important components of a
thesis, to give explanations and/or conclusions.  If a figure or table is
quoted, indicate its source.
\item
Draw figures and tables carefully and accurately, taking account of the
appropriateness of symbols in them.  Give a good heading, which is a title
or a short explanation, to each figure and table.
\item
Cite selected important literatures that are tightly related to your work,
instead of making a lengthy and redundant bibliography.  The list of
references should be placed just after the main text (just before
appendices, if any).
\end{itemize}%}

\subsubsection{Conclusions}\label{subsubsec-conclusion}
The concluding section summarizes the main results of your research work.
Thus this part is naturally the essence of the main issue and concludes the
thesis in a more lofty style than the abstract.

If you found derivative and/or unsolved subproblems that are left for
future works, they should be mentioned in the concluding section.

The authors of papers traditionally express their thanks to the people who
give them guidance, suggestions and supports on their research works.  Thus,
it is strongly recommended to follow the tradition and to record your
acknowledgments to such people after the conclusions.

\subsection{Appendices}\label{subsec-appendix}
In principle, a thesis must be completed by the components shown above.
However, you may add appendices to supplement the main text.

The following are examples that should be moved to appendices, and advice
on compiling appendices.
\begin{itemize}%{
\item
Long and complicated formula derivations, which are hardly included in the
main text because of their length and readability.  Give cross references
both in the main text and the appendix to show the correspondence between
them.
\item
Proofs of theorems, similarly.  If the objective of your research work is to
prove these theorems, however, the proofs (or simplified versions, at least)
should be in the main text.
\item
Huge results of observations and/or numerical computations.  Although such a
result that supports arguments in a thesis is preferably in the main text,
huge one should be moved in an appendix leaving a summary in the main text.
\item
Long source program lists.
\item
If an extremely huge data or a program list is really necessary to be shown,
it may be attached as a separate volume.  In this case, the separate volume
should have a form similar to the main volume for the convenience of keeping
and referring.
\item
Each appendix, especially for that in a separated volume, should has an
appropriate title and a brief explanation so that readers grasp what the
appendix shows.
\end{itemize}%}

\section{Instructions on Writing Theses}\label{sec-instruction}
\subsection{Languages}\label{subsec-language}
A graduation research report should be written in {\EM Japanese}.

A master thesis may be written in either {\EM Japanese or English}.

Since you will have many opportunities to present your research works in
international conferences, writing your thesis in English might be a good
training.  On the other hand, writing a good and correct Japanese thesis is
also important for a student who will work in Japan.  Therefore, the
Department judged that writing a report in Japanese is appropriate and
reasonable for a undergraduate student.

Regardless of language that you choose for the main text, {\EM both
Japanese and English abstracts} must be given, as described in
Section~\ref{subsec-abstract}.

Technical terms must be spelled correctly.  Good examples of technical terms
will be found in transactions and journals on your research area.  If you
cannot find a standard translation of a term, use the original word, or
write a non-standard translation with the original word.  A proper noun
should be spelled in the original word or in Japanese Katakana.  In the
latter case, it is recommended to show the original word at its first
appearance.

\subsection{Symbols and Units}\label{subsec-symbol}
It is strongly recommended that a thesis with a lot of mathematics is
typeset using \LATEX{} or other tool that is good at math handling.  If
you have to use an ordinary word-processor, choose fonts for math formulae
carefully and pay close attention to the positions of subscripts and
superscripts.

Symbols used in your thesis must be defined clearly, as mentioned before.
If you use many symbols, you may define them in a ``table of symbols''
inserted in an appropriate place.

Abbreviated symbols for measuring units should be those standardized
and\slash or used in transactions and journals widely.  As for a unit whose
standard symbol has not been fixed, show what it stands for (in a footnote,
for example).

\subsection{Figures and Tables}\label{subsec-figure}
Figures and tables must be inserted in the main text.  Arrange the position
of a figure or a table so that it is as close as possible to the text that
refers it.

The number of a table, like ``Table~1.3'' and its heading must appear above
the table.  In case of a figure, its number, such as ``Figure 2.1'', and its
heading must be placed below the figure.

Use some computer-aided drawing tool, if available, for figures.  Otherwise,
hand-written figures must be drawn with very much care, as if they are used
to make typographical plates.  Note that pencil-drawn figures are not
acceptable.

In general, a figure or a table must be self-understandable.  On the other
hand, some description of a figure/table must appear in the main text with
the reference of its number.

If you want to show huge results of observations and/or numerical
computations, move them to an appendix leaving some important ones in the
main text.

\subsection{Footnotes}\label{subsec-footnote}
Footnotes are {\em not} recommended.  It is occasionally acceptable,
however, to use a footnote for a short annotation in order to make the main
text less complicated.

A footnote must have a number to show its correspondence to a part of the
main text, as shown in the example below.

A footnote may also be used to cite a literature describing, for example, a
proof that is not a part of the main arguments of a thesis.  However,
remember this citation style is quite exceptional and the recommended style
is to use a bibliography list that is explained later.

\begin{description}
\item[Example {\rm(Main Text)}]\leavevmode\par
\ldots this sequential approximation method is used in the proof of a theorem
that state differential equations has solutions\footnote{This concept is
introduced by Picard, et al. (1890).}, and \ldots\ is known as a general
feature\footnote{This proof is given in ``Modern Analysis'' by Whittaker and
Watson, p.~123.}.
\end{description}

\subsection{Literatures}\label{subsec-references}
A list of cited literatures should be compiled and should be placed just
after the main text (and just before the appendices, if any).  Each item of
the list has a bracketed serial number, like [1], [2], etc.  In the main
text, the number also appears at the end of the phrase in which
corresponding literature is referred.  A word, such as the name of a person,
may also be followed by the number.

For each literature in the list, the name of its author(s), its title, the
name of the journal in which it appears, volume number and pages, and
published year are recorded, as shown in the bibliography list of this guide.
The literatures [1] to [3] appear in journals and proceedings, while [4] is
a book.

If a journal has a well-known abbreviation, its full name can be replaced
with the abbreviation.  In the list of this guide, the article [2]
appears in ``IEEE Trans. Computers'' whose full name is ``IEEE Transactions
on Computers''.

For an article in a journal, it is recommended to add the issue number and
published month of the journal together with the volume number and published
year.  The following is an example of the issue number and published month.
\begin{eqnarray*}
&\hbox{The tenth issue in volume four of a journal, published on October 1995.}
\\
&\Big\Downarrow\\
&\hbox{Vol.~4, No.~10 (Oct.~1995)}.
\end{eqnarray*}

\subsection{Printed Form and Pages}\label{subsec-format}
A thesis must be printed on single sides of A4 form papers.  {\EM The main
text including figures and tables} must have the following pages, allowing
10\,\% excess or shortage.
\begin{itemize}%{
\item
Graduation research report\ \dotfill\ 25 pages\hbox to10em{}
\item
Master thesis\ \dotfill\ 50 pages\hbox to10em{}\\
\hbox{}\hfill(a thesis written in English may have 60 pages)
\end{itemize}%}
Figures and tables may occupy up to about 40\,\%, and should be moved to
appendices if exceeded.

A thesis must be printed by a typesetting tool, like \LATEX, or a
word-processor.  Note that {\EM hand-written theses will not be accepted}.

Each page must have {\EM 3\,cm left margin and 1\,cm right margin} at least.

For a thesis written in Japanese, use 12\,pt (or similar size) fonts to form
{\EM lines of 35 characters and pages of 32 lines}.  If you use a
word-processor and have troubles to keep this line/page format, you may
change characters per line and lines per page unless the number of
characters in a page are not (greatly) changed.

For a thesis written in English, use 12\,pt (or similar size) fonts to form
{\EM lines of 14.2\,cm wide and pages of 32 lines}.

Regardless of language in which your thesis is written, give two lines to a
top or second level section heading, while one line to a third or lower
level.  Do not insert empty lines nor vertical spaces above/below a
itemization nor between items in it.

\subsection{Cover, Title Page and Table of Contents}\label{subsec-title}
A title page must be attached at the very beginning of a thesis (i.e. just
before the abstract).  Whole thesis must be bound up in the binder that the
Department designates.

On the cover of the binder, paste a cover sheet to show the type of your
thesis (i.e. B.~E or M.~E thesis), its title, your supervisor name, the name
of Department that you belong, your name, and submission date.  The
same items must be printed on the title page too.

At the top of both Japanese and English abstract pages, show the title of
your thesis and your name.  The title must appear at the top of the first
page for table of contents.

\subsection{Submission}\label{subsec-submission}
Complete your thesis and submit it to the Department office no latter than
the date that will be announced beforehand.  Keep the deadline strictly.

\subsection{Other Suggestions}\label{subsec-others}
Before you start writing your thesis, take enough time to plan its structure
and length.  Discussion on the contents of your thesis in your laboratory is
also required.  Before finishing, proofread your thesis again and again by
yourself to make the thesis free from logical leaps and inconsistency.  The
proofreading is also required to polish sentences and to remove typos.  It
is recommended to ask senior members in your laboratory to correct errors
caused by your misunderstanding and/or carelessness.

\section{Concluding Remarks}\label{sec-conclusion}
This guide described how to structure your thesis and how to form it.
However, this guide is far from a fully-automatic procedure to write a good
thesis, as mentioned at the very beginning.

The most important thing that you must have in your mind on writing your
thesis is a strong will to let readers understand your research work.
Having enthusiasm for improvement your thesis is also important.  We hope
you conclude your research work with strong will and enthusiasm.

Finally, if you have any questions on writing your thesis, ask your
supervisor.

\acknowledgments				% Acknowledgments
The author of this guide would like express his thanks to all the teaching
staffs in the Department of Information Science for their
contribution.

\nocite{*}
\bibliographystyle{kuisunsrt}			% BibTeX style
\bibliography{eguide}				% Output references

						% Here an appendix starts
\Appendix[Appendix: How to Use Style File {\tt kuisthesis}]
In order to produce a well-formatted thesis following the guidelines and
instructions given in this guide, a \LaTeX{} style file, named \|kuisthesis|
is provided.  This appendix explains what you have to do before use the style
file and how to use it.

Since this guide itself is produced with the style file, it will help you to
refer its source file \verb|eguide.tex| as an example.

Most of \LaTeX{} commands that will be used in your thesis source file are
standard ones.  Therefore, see the following book or other textbooks for
basic usage of commands and \LaTeX{} features not explained here.
\begin{quote}%{
Lamport, L.: {\em A Document Preparation System {\LaTeX} User's Guide \&
Reference Manual\/}, Addison Wesley, Reading, Massachusetts (1986).
\end{quote}%}

\par\bigskip\centerline{\bf NOTE}
\begin{quote}
In the following explanation, it is assumed that you intend to write your
thesis in English.  Otherwise, see the Japanese version of the guide.
\end{quote}

\section{Preliminaries}\label{app-prelim}
\subsection{Obtaining Thesis Kit}\label{appsub-kit}
The style file \|kuisthesis| and other related files are packaged in a
\|tar|'ed and \|gzip|'ed file;
\begin{itemize}\item[]\small%{
\|ftp://ftp.kuis.kyoto-u.ac.jp/ku/kuis-thesis/kuisthesis.tar.gz|\quad.
\end{itemize}%}

This package consists of the following files:
\begin{itemize}%{
\item
\|kuisthesis.sty|\,:
Style file.
\item
\|kuisthesis.cls|\,:
Style file for \LATEXe.
\item
\|kuissort.bst  |\,:
Bib\TeX{} style file (sorted).
\item
\|kuisunsrt.bst |\,:
Bib\TeX{} style file (unsorted).
\item
\|eguide.tex    |\,:
Source file of this guide. 
\item
\|guide.bib     |\,:
Bib\TeX{} database file for this guide.
\item
\|guide.tex     |\,:
Source file of Japanese version.
\item
\|guide.bib     |\,:
Bib\TeX{} database file for Japanese version.
\end{itemize}%}

\subsection[{\protect\LaTeX} Environment]{{\protect\LATex} Environment}
\label{appsub-env}
Although your thesis is written in English, an abstract in Japanese must be
included in it.  Therefore, you have to use one of Japanese {\LaTeX}
systems, that based on j{\TeX} (or NTT version) developed by Saito of NTT,
or on Japanese {\TeX} (or ASCII version) provided from ASCII corporation.
Since the style file cope with both versions, you may use what is available
to you.

The style files are confirmed to be worked with the following versions.
\begin{itemize}%{
\item
NTT${}={}${j\TeX} 1.52${}+{}${\LaTeX} 2.09
\item 
ASCII${}={}${\TeX} 2.99-j1.7${}+{}${\LaTeX} 2.09
\end{itemize}%}
Although we expect the style file will work with older versions, it is
strongly recommended to use the versions shown above.

As for {\LATEXe}, the style file is workable with the following versions.
\begin{itemize}%{
\item
NTT${}={}${j\TeX} 1.6${}+{}$%
	{\LATEXe} 1994/12/01 patch level 3
\item 
ASCII${}={}${p\TeX} 3.1415 p2.1.4${}+{}$%
	{p\LATEXe} 1995/09/01
\end{itemize}%}
You may use the styles in either native-mode or {\LaTeX} 2.09 compatible
mode.

\section{Configuration of Source File}\label{app-structure}
A source file must have the following format.
\begin{itemize}\item[]\it%{
\|\documentstyle[master,english]{kuisthesis}|\footnote{%
Replace it with {\CS\documentclass} and, if necessary, add {\CS\usepackage}
when you use {\LATExe} in native mode.}\\
\null\qquad Specify other options/styles if necessary.\\
Define your own macros, etc., if necessary.\\
\|\jtitle{|\<title in Japanese\>\|}|\\
\|\etitle{|\<title in English\>\|}|\\
\|\jauthor{|\<author's name in Japanese\>\|}|\\
\|\eauthor{|\<author's name in English\>\|}|\\
\|\supervisor{|\<supervisor's name\>\|}|\\
\|\date{|\<submission date\>\|}|\\
\|\department{|\<department name\>\|}|\\
\|\begin{document}|\\
\|\maketitle|\hfill\rlap{\hskip-.5\linewidth{\tt\%} output title page}\\
\|\begin{eabstract}|\\
\null\qquad\<abstract in English\>\\
\|\end{eabstract}|\\
\|\begin{jabstract}|\\
\null\qquad\<abstract in Japanese\>\\
\|\end{jabstract}|\\
\|\tableofcontents|\hfill
	\rlap{\hskip-.5\linewidth{\tt\%} output table of contents}\\
\|\section{|\<1st section heading\>\|}|\\
\null\qquad\hbox to3em{\dotfill}\\
\null\qquad\<main text\>\\
\null\qquad\hbox to3em{\dotfill}\\
\|\acknowledgments|\\
\null\qquad\<acknowledgments\>\\
\|\bibliographystyle{kuisunsrt}|\quad or\\
\|\bibliographystyle{kuissort}|\\
\|\bibliography{|\<Bib\TeX{} database file\>\|}|\\
Put appendices here following \|\appendix| or \|\Appendix|, if any.\\
\|\end{document}|
\end{itemize}%}
Each component is explained in the following sections.

\subsection{Page Format}\label{appsub-format}
Each page of a thesis is printed in a area whose width (\|\textwidith|) is
14.2\,cm and height (\|\textheight|) is 22.2\,cm.  The hight is just enough
to contain 32 lines, which is the regulation of a page shown in
Section~\ref{subsec-format}.

Fonts in \|\normalsize| is the size of 12\,pt, which also satisfies the
regulation.

\subsection{Option Styles}\label{appsub-option}
The following three style options are available to be specified in the
optional argument of \|\documentstyle| (or \|\documentclass|).
\begin{itemize}%{
\item
\|master|\\
For a master thesis.  You must specify it because you intend to write a
master thesis.  Otherwise, your thesis will be regarded as a graduation
research report mistakingly.
\item
\|english|\\
For a thesis written in English, which you must be on writing.  If omitted,
the result will have stuff in Japanese, which you do not want to have.
\item
\|withinsec|\\
Produce numbers of figures, tables and math formulae in the form of
``\<S\>.\<N\>'' where $S$ is the \|\section| number in which a figure etc.\
appears, and $N$ is its serial number in a \|\section|.  If omitted,
numbering will runs through the whole of a thesis.
\end{itemize}%}

In the optional argument of \|\documentstyle| (or the mandatory argument of
\|\usepackage|), you may specify supplementary style files such as \|epsf|.
Note that style files may be incompatible to the style of theses.  For
example, you cannot use \|a4| style because it modifies the height of a page
unexpectedly.

\subsection{Specifying Title, etc.}\label{appsub-title}
The title of your thesis, you name, and your supervisor's name have to be
defined by appropriate commands shown at the beginning of this section.
After the definitions, do \|\maketitle| to make a title page.

In the title page, the following items are centered and printed in order.
\begin{description}%{
\item[type of thesis]
Since you specified \|master| and \|english| option in the argument of
\|\documentstyle|, the type of your thesis, ``Master Thesis'', is printed
using \|\Large\bf| font.

\item[title]
The title specified by \|\etitle| is printed using \|\LARGE\bf| font.  If
the title is long, put \|\\| in the title string to show a line break point,
instead of depending on automatic line breaking.

The title is also put on the pages of abstract and table of contents.  Since
your thesis must have a Japanese abstract with a Japanese title, give the
Japanese title using \|\jtitle|.  Desirable line break points of the title
in abstract and table of contents may be different from those in the title
page.  To cope this, \|\etitle| can have an optional argument to specify the
title string for abstract and table of contents as;
\begin{quote}%{
\|\etitle[|\<opt-title\>\|]{|\<title\>\|}|
\end{quote}%}
where \<opt-title\> is the title for abstract and table of contents, while
\<title\> is put in the title page.  For example, the title of this guide is
specified by;
\begin{quote}\begin{verbatim}
\etitle[How to Write Your B.E./M.E. thesis]%
       {How to Write//Your B.E./M.E. thesis}
\end{verbatim}\end{quote}
to force line break only in the title page.

\item[supervisor name]
The name of your supervisor and his/her title (e.g. ``Professor''), which
are specified by \|\supervisor|, are printed using \|\large| font.

\item[affiliation]
You certainly are a master course student and thus belong to ``Department of
Information Science, Graduate School of Engineering, Kyoto University''.
Therefore, this string is printed using \|\large| font.

\item[author name]
Your name given in the argument of \|\eauthor| is printed using \|\Large|
font.  Similar to the title of your thesis, your name also appear in the
abstract pages and you must show your name with Japanese characters using
\|\jauthor| command.

\item[submission date]
The command \|\date| is to specify the date on which you submit your thesis.
The date is printed using \|\large| font.

\item[department name]
You should specify the name of department like
\begin{quote}\begin{verbatim}
\department{Social Informatics}
\end{verbatim}\end{quote}
to print it in the title page.

\end{description}%}
In the title page, its page number is not printed.  However, the page in
\|dvi| file has an imaginary page number 1000 for convenience of printing
procedure.

\subsection{Abstract}\label{appsub-abstract}
An abstract in English should be given in \|eabstract| environment, while
its Japanese counterpart \|jabstract| is for a Japanese abstract.  At the
top of the first page of each abstract, the title given by
\|\etitle|\slash\|\jtitle| and your name specified by
\|\eauthor|\slash\|\jauthor| are printed.

The order of printing two abstracts is that of your source file.  Thus, it is
natural to put \|eabstract| environment before \|jabstract|.

The page number of each abstract page is printed at the right-upper corner in
lowercase Roman numeral.  Although the page numbers are ``i'', ``ii'' and so
on, naturally, 1000 is added to those in \|dvi| file to distinguish them
from pages of the main text in printing.

\subsection{Table of Contents}\label{appsub-toc}
A command \|\tableofcontents| produce a table of contents.  At the top of
its first page, the title specified by \|\etitle| is printed.

In default, the table contains headings and page numbers of \|\section|,
\|\subsection| and \|\subsubsection|.  If you want to eliminate
\|\subsubsection|, for example, do;
\begin{quote}\begin{verbatim}
\setcounter{tocdepth}{2}
\end{verbatim}\end{quote}
to set the counter \|tocdepth| to 2 so that only level-1 (\|\section|)
and level-2 (\|\subsection|) sections appear in the table.

Additionally, ``Acknowledgments'' and ``References'' are included in the
table as numberless \|\section|.  If your thesis has an appendix,
``Appendix'' and its \|\section| and \|\subsection| are also included.

Page numbers are not printed for the table of contents, but \|dvi| file
pages have 1000-plus numbers following abstract pages.

\subsection{Sectioning}\label{appsub-sectioning}
As in usual \LaTeX{} documents, sectioning is done by \|\section|,
\|\subsection|, \|\subsubsection| and so on.

A heading given by \|\section| is in two lines and is
printed using \|\Large\bf| font.  A \|\subsection| heading is printed using
\|\large|\allowbreak\|\bf| font, leaving one blank line above it and no
extra spaces below it.  The command \|\subsubsection| acts as \|\subsection|
but no blank lines are not inserted above and the font size is \|\normalsize|.

In default, a heading given by one of these three commands will have a
section number, while lower level commands \|\paragraph| and
\|\subparagraph| will not show section numbers.  Those lower level commands
will not break lines after headings specified by them.

\subsection{Figures and Tables}\label{appsub-figure}
Figures and tables are contained in \|figure| and \|table| environments,
respectively, as usual.  Figure (or table) numbers run through the whole of
main text in default.  If you use \|withinsec| option of \|\documentstyle|,
however, numbers runs only through a \|\section| and are shown combined with
a \|\section| number.

Sometimes, you may want to line up two or more figures and/or tables
horizontally to saving space.  Two environments \|subfigure| and \|subtable|
aims at this horizontal placement.  For example, Figure~\ref{fig-example} and
Table~\ref{tab-example} are produced by the following command sequence.
\begin{quote}%{
\|\begin{figure}|\\
\|\begin{subfigure}{0.6\textwidth}|\\
\null\qquad\<body of Figure~\ref{fig-example}\>\\
\|\caption{Example of Figure}|\\
\|\end{subfigure}|\\
\|\begin{subtable}{0.4\textwidth}|\\
\|\caption{Example of Table}|\\
\null\qquad\<body of Table~\ref{tab-example}\>\\
\|\end{subtable}|\\
\|\end{figure}|
\end{quote}%}
In this example, \|subfigure| and \|subtable| are enclosed by \|figure|
environment.  However, you may enclose them by \|table|.

The environments \|subfigure| and \|subtable| have the following
specifications, similar to \|minipage|.
\begin{itemize}\item[]%{
\|\begin{subfigure}[|\<position\>\|]{|\<width\>\|}|\quad\<body\>\quad
\|\end{subfigure}|\\
\|\begin{subtable}[|\<position\>\|]{|\<width\>\|}|\quad\<body\>\quad
\|\end{subtable}|
\end{itemize}%}
The command \|\caption| in an environment produce a heading appropriate to
the environment.

It is recommended that the sum of widths of \|subfigure|\slash\|subtable| is
equal to \|\textwidth|\footnote{You may put spaces between them by, for
example, {\tt\string\hspace\char`\{\string\fill\char`\}}}.

\begin{figure}%{
\begin{subfigure}{.6\textwidth}
\centerline{\fbox{\vbox to.1\textheight{\vss
	\hbox to.8\textwidth{\hss This is a figure\hss}\vss}}}
\caption{Example of Figure}\label{fig-example}
\end{subfigure}
\begin{subtable}{.4\textwidth}
\caption{Example of Table}\label{tab-example}
\centerline{\begin{tabular}{r|c|l}
This&is&a table\\\hline
placed&beside&a figure.
\end{tabular}}
\end{subtable}
\end{figure}%}

\subsection{Itemizing}\label{appsub-itemizing}
Standard \LaTeX{} itemizing environments, \|enumerate|, \|itemize|,
\|description| and so on, are available as usual.  Note that no extra
vertical spaces are inserted above\slash below an environment nor between
two \|\item|s.

\subsection{Footnotes}\label{appsub-footnote}
A \LaTeX{} standard command \|\footnote| will make a footnote.  A footnote
marks is this\footnote{An example of footnote.} or this\footnote{Another
footnote example.} or those like them.  Observing this page and pages having
footnotes, you will find that footnote numbers runs only through a page.
Remember that you need to run \LaTeX{} twice to have correct footnote
numbers.

\subsection{Acknowledgments}\label{appsub-acknowlegments}
Your acknowledgments must follow a command \|\acknowledgments|.  The heading
``Acknowledgments'' is automatically produced and is included in the table
of contents.

\subsection{Bibliography}\label{appsub-references}
Use Bib\TeX{} to make a list of references easily.  First, make a database
file that includes bibliographic informations of all the literatures cited in
your thesis.  Then process it with a bibliographic style \|kuisunsrt| or
\|kuissort| to have a well-formatted list like that shown in
Section~\ref{subsec-references} and the list of this guide.  The order of
items in a list is that of appearance if you use \|kuisunsrt|, while
\|kuisort| sorts items alphabetically using author names as keys.

If you cannot use Bib\TeX, you may make the list by \|thebibliograhpy|
environment, carefully following the style of the list in this guide.

Whether you use Bib\TeX{} or \|thebibliography|, the heading ``References'' is
automatically produced and is included in the table of contents.

\subsection{Appendices}\label{appsub-appendix}
If your thesis has an appendix, it must follows a command \|\appendix| or
\|\Appendix|.  These two commands act similarly, except for page numbering.
The command \|\appendix| inhibits page number printing on each page nor each
line of the table of contents.  On the other hand, \|\Appendix| will produce
appendix pages numbered A-1, A-2, and so on.  In both cases, pages in \|dvi|
file have imaginary page numbers from 2001.

Both commands have an optional argument to specify the heading of an
appendix itself.  For example, this appendix starts with
\begin{itemize}\item[]
\|\Appendix[Appendix: How to Use Style File {\tt kuisthesis}]|\quad.
\end{itemize}
If the optional argument is not given, the heading of an appendix is simply
``Appendix''.

Each sectioning command in an appendix pretends as if it is ranked one level
down.  For example, a \|\section| acts as a \|\subsection|.  Each section
number has a prefix ``A.'', and thus it will be ``A.1'' or ``A.2.3'' or like
them.  Similarly, the number of a figure, a table or a math formula also has
the prefix ``A.''\footnote{This prefixing is independent from {\tt withinsec}
option of {\tt\string\documentstyle}.}.

\section{Other Suggestions}\label{app-others}
One of the important feature of \LaTeX{} is its capability of customizing
functions and parameters for document processing.  Thus, it is not forbidden
but rather recommended that you add and devise your own functions in order
to help yourself in writing a thesis.  On the other hand, your own functions
must not break the thesis format specified by the Department.  Thus, it is
necessary to balance innovation with conservation.

There are no simple and general guideline to judge whether a customization
is allowed.  One extreme guideline is that you do something only if you have
confidence to do it after fully understanding the style file.  If you feel
proud that you are a \LaTeX{}nician, follow this guideline.

For other students, a not-so-extreme guideline might be that existing
commands and parameters cannot be redefined.  If you feel reading the
style file to be troublesome, or you cannot understand the style file,
follow this guideline.

The author of the style file made all possible efforts in order that the
style file is bug-free.  However, the author must confess that the style
file may not be perfect, mainly because of lack of much application
experience.  If you have any trouble on using the style file, feel free to
post your problem to the local news group \|is.misc| of the Department.
An announcement of new version release and other important notifications will
be posted to the news group too.  Remember that a teaching staff who is
responsible for maintenance of the style file will {\em NOT} accept direct
inquiries to him/her.
\end{document}
